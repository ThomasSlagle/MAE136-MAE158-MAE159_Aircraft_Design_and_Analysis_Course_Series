\documentclass{article}
\usepackage{geometry}
\geometry{margin=1in}
\usepackage{multirow}
\usepackage[table,xcdraw]{xcolor}
\usepackage{stfloats}
\usepackage{setspace}

\title{MAE 159 Midterm Aircraft Sizing Report}
\author{Thomas Slagle}
\date{May 7th, 2021}

\parskip=10pt

\begin{document}
    \maketitle

    \section{Introduction}
    \begin{flushleft}
        This report consists of a study on the cost and performance optimization
        for two subsoinc commercial transport aircraft. Herein, the reader will
        find a summary of the methods used and the data generated from an
        itterative python script which uses standard, well-defined aircraft
        deisgn methods to exactly meet the design specifications. Various
        parameters, including ... , were systematically varied to determine the
        optimum design parameters.
    \end{flushleft}

    \section{Design Specifications}
    \begin{flushleft}
        As mentioned prior, two aircraft with distinct given design
        requirements, were considered in this design study. Both aircraft are
        requried to carry 225 passangers adn complete a 7400 nautical mile
        journey. The first larger aircraft must compelte the journey without any
        stops. The second smaller aircraft must complete the journey with
        one-stop, giving the airplane a required range of 3700 nautical miles.
        The complete set of given design specifications are listed in tables 1
        and 2 below. For both aircraft, takeoff conditions were assumed to be at
        sea level on a hot day with an air temperature of $84^{\circ}F$.
    \end{flushleft}

\begin{table}[ht]
    \begin{tabular}{|c|c|}
    \hline
    \rowcolor[HTML]{FFC702}
    \multicolumn{2}{|c|}{\cellcolor[HTML]{FFC702}\textbf{Non-stop Aircraft}} \\ \hline
    \textbf{Design Specification:}        & \textbf{Parameter Value:}        \\ \hline
    Number of Passangers                  & 225                              \\ \hline
    \rowcolor[HTML]{C0C0C0}
    Weight of Cargo                       & 6,000 lbs                        \\ \hline
    Still Air Range                       & 7,400 nmi                        \\ \hline
    \rowcolor[HTML]{C0C0C0}
    Takeoff Field Length                  & 10,500 ft                        \\ \hline
    Landing Approach Speed                & 140 kts                          \\ \hline
    \rowcolor[HTML]{C0C0C0}
    Fuel Destination Payload              & 35\%                             \\ \hline
    Cruise Mach Number                    & 0.85                             \\ \hline
    \rowcolor[HTML]{C0C0C0}
    Initial Cruise Altitude               & 35,000 ft                        \\ \hline
    \end{tabular}
    \quad
    \begin{tabular}{|c|c|}
        \hline
        \rowcolor[HTML]{DAE8FC}
        \multicolumn{2}{|c|}{\cellcolor[HTML]{DAE8FC}\textbf{One-stop Aircraft}} \\ \hline
        \textbf{Design Specification:}        & \textbf{Parameter Value:}        \\ \hline
        Number of Passangers                  & 225                              \\ \hline
        \rowcolor[HTML]{C0C0C0}
        Weight of Cargo                       & 3,000 lbs                        \\ \hline
        Still Air Range                       & 3,700 nmi                        \\ \hline
        \rowcolor[HTML]{C0C0C0}
        Takeoff Field Length                  & 6,000 ft                         \\ \hline
        Landing Approach Speed                & 130 kts                          \\ \hline
        \rowcolor[HTML]{C0C0C0}
        Fuel Destination Payload              & 0\%                              \\ \hline
        Cruise Mach Number                    & 0.80                             \\ \hline
        \rowcolor[HTML]{C0C0C0}
        Initial Cruise Altitude               & 35,000 ft                        \\ \hline
    \end{tabular}
\end{table}




\end{document}
