\documentclass{article}
\usepackage{geometry}
\geometry{margin=1in}
\usepackage{multirow}


%Figures
\usepackage{subcaption}
\usepackage{graphicx}
\usepackage{array}
\graphicspath{ {./images/} }

%Tables
\usepackage{longtable}
\usepackage[table,xcdraw]{xcolor}
\usepackage{stfloats}
\usepackage{setspace}


\title{MAE 159 Midterm Airplane Design Final Report}
\author{Thomas Slagle (72602272)}
\date{June 4th, 2021}

\parskip=8pt
\setlength{\textfloatsep}{8pt plus 1.0pt minus 2.0pt}

\begin{document}
    \maketitle
    \begin{center}
        \includegraphics[scale=1.0]{Aircraft.PNG}

        \vspace{1in}

        \includegraphics[scale=0.25]{pngjoy.com_uci-logo-uc-irvine-henry-samueli-school-of_6860660.png}
    \end{center}

    \pagebreak
    \tableofcontents
    %\pagebreak
    %\listoffigures
    %\listoftables
    \pagebreak

    \section{Introduction}
    \label{sec:intro}
    \begin{flushleft}
        This report consists of a study on the cost and performance optimization
        for two subsoinc commercial transport aircraft with a design range of
        7400 nautical miles. One aircraft must meet the design range without any
        stops and the other aircraft is designed to make one stop in transit.
        These aircraft will herein be referred to as the non-stop and one-stop
        aircraft respectively. In this report, the reader will find a summary of
        the methods used and the data generated from an iterative python script
        which uses standard, well-defined aircraft design methods to exactly
        meet the design specifications on each iteration. Various parameters,
        including the aspect ratio, the airfoil type, the wing sweep angle, and
        the seating configuration, were systematically varied to determine the
        optimum design parameters for each aircraft. After optimization of the
        basic aircraft properties, the aircraft was configured with detailed
        interior and exterior arrangements. The conclusion of this report
        provides customer recommendations and a fuel burn analysis.
    \end{flushleft}

    \section{Design Specifications}
    \label{sec:specs}
    \begin{flushleft}
        As mentioned prior, two aircraft with distinct given design
        requirements, were considered in this design study. Both aircraft are
        required to carry 225 passengers and complete a 7400 nautical mile
        journey. The complete set of given design specifications are listed in
        tables 1 below. For both aircraft, takeoff conditions were assumed
        to be at sea level on a hot day with an air temperature of
        $84^{\circ}F$.
    \end{flushleft}

    \begin{table}[ht]
        \begin{tabular}{|c|c|}
        \hline
        \rowcolor[HTML]{FFC702}
        \multicolumn{2}{|c|}{\cellcolor[HTML]{FFC702}\textbf{Non-stop Aircraft}} \\ \hline
        \textbf{Design Specification:}        & \textbf{Parameter Value:}        \\ \hline
        Number of Passengers                  & 225                              \\ \hline
        \rowcolor[HTML]{C0C0C0}
        Weight of Cargo                       & 6,000 lbs                        \\ \hline
        Still Air Range                       & 7,400 nmi                        \\ \hline
        \rowcolor[HTML]{C0C0C0}
        Takeoff Field Length                  & 10,500 ft                        \\ \hline
        Landing Approach Speed                & 140 kts                          \\ \hline
        \rowcolor[HTML]{C0C0C0}
        Fuel Destination Payload              & 35\%                             \\ \hline
        Cruise Mach Number                    & 0.85                             \\ \hline
        \rowcolor[HTML]{C0C0C0}
        Initial Cruise Altitude               & 35,000 ft                        \\ \hline
        \end{tabular}
        \quad
        \begin{tabular}{|c|c|}
            \hline
            \rowcolor[HTML]{DAE8FC}
            \multicolumn{2}{|c|}{\cellcolor[HTML]{DAE8FC}\textbf{One-stop Aircraft}} \\ \hline
            \textbf{Design Specification:}        & \textbf{Parameter Value:}        \\ \hline
            Number of Passengers                  & 225                              \\ \hline
            \rowcolor[HTML]{C0C0C0}
            Weight of Cargo                       & 3,000 lbs                        \\ \hline
            Still Air Range                       & 3,700 nmi                        \\ \hline
            \rowcolor[HTML]{C0C0C0}
            Takeoff Field Length                  & 6,000 ft                         \\ \hline
            Landing Approach Speed                & 130 kts                          \\ \hline
            \rowcolor[HTML]{C0C0C0}
            Fuel Destination Payload              & 0\%                              \\ \hline
            Cruise Mach Number                    & 0.80                             \\ \hline
            \rowcolor[HTML]{C0C0C0}
            Initial Cruise Altitude               & 35,000 ft                        \\ \hline
        \end{tabular}
        \caption{Given Design Specifications.}
    \end{table}

    \section{Design Analysis}
    \label{sec:design}
    \begin{flushleft}
        The object of this section is to perform an analysis for both aircraft
        to determine the optimized specifications for the given design
        parameters. An iterative python script was developed with
        user-selectable design parameters to make calculations of direct
        operating cost (DOC), weight, drag, and other aircraft performance
        characteristics easy, fast, and repeatable.
    \end{flushleft}

    \subsection{Direct Operating Cost versus Aspect Ratio and Wing Sweep Angle}
    \label{sec:AR}
        \begin{flushleft}
            The aspect ratio describes the ratio of the aircraft's wingspan to
            its mean aerodynamic chord length. A small aspect ratio describes a
            short and wide wing whereas a larger aspect ratio describes a long
            and narrow wing planform (1). The wing aspect ratio is an important
            factor in determining the available lift of the aircraft, the weight
            of the aircraft, and the induced drag during flight. For a typical
            jet transport aircraft, Schaufele gives an aspect ratio range of 7.0
            to 9.5 (3), as such, this formed the basis for design selection.
            Aspect ratios in steps of 0.1 were considered from 6.0 to 12.0
            during this study. The method for comparison will be the resulting
            DOC per passenger, per mile. Figure \ref{fig:doctmAR} shows the
            aspect ratio versus the DOC with curves of fixed sweep angle for
            both the short range and long range aircraft. A total of 1800
            discrete aircraft designs were considered. The plots in
            \ref{fig:doctmAR} show every 5 degrees of wing sweep angle from 10
            to 40 degrees, along with the optimized wing sweep angle, which
            varied for both aircraft. These designs also consider two JT9D
            engines with one aisle, six abreast for the seating configuration.
        \end{flushleft}

        \begin{center}
            \includegraphics[scale=0.50]{DOCTM v Sweep Angle.PNG}
            \captionof{figure}{Direct operating cost, per ton, per mile plotted against aspect ratio for the non-stop and one-stop aircraft at different wing sweep angles.}\label{fig:doctmAR}%      only if needed
        \end{center}

        \begin{flushleft}
            From figure \ref{fig:doctmAR}, the optimized aspect ratio can be
            determined by finding the range at which the minimum value occurs.
            Although 1800 designs were considered, the sweep angle curves are
            only plotted for every 5 degrees and for the optimum angle, to allow
            the figure to be more discernable. From this analysis, it is evident
            that for the non-stop aircraft, the optimized aspect ratio is 7.9
            with a wing sweep angle of 37 degrees. For the one-stop aircraft,
            the optimized aspect ratio is 9.5 with a wing sweep angle of 31
            degrees. This corresponds to a direct operating cost, per ton, per
            mile of \$0.1064 and \$0.1026 for the one-stop and nonstop aircraft
            respectively. The rest of the sections within section
            \ref{sec:design} will be based off of the control values found in
            this section of the report, using conventional technology and a
            seating configuration of one aisle and six seats abreast.
        \end{flushleft}

    %\pagebreak
    \subsection{Weight versus Aspect Ratio and Wing Sweep Angle}
    \label{sec:weight}
        \begin{flushleft}
            Now that the most efficient plane in terms of cost to operate has
            been found, it is of interest to look at the weight of each discrete
            aircraft design generated in section \ref{sec:AR}. Figure
            \ref{fig:AR} plots the takeoff weight of the aircraft in pounds
            versus the aspect ratio using fixed lines of sweep angle, using the
            same plotting scheme as the previous section. The plots in this
            figure show a much more dramatic divergence from the optimum value
            than the direct operating cost plots. The minimum weight values
            occur near the range of aspect ratios for the lowest direct
            operating cost.
        \end{flushleft}

        \begin{center}
            \includegraphics[scale=0.5]{Weight v Sweep Angle.PNG}
            \captionof{figure}{Aircraft takeoff weight plotted against sweep angle for the non-stop and one-stop aircraft at the optimized aspect ratios.}\label{fig:AR}
        \end{center}

    %\pagebreak
    \subsection{Advanced Technology}
    \label{sec:adv_tech}
        \begin{flushleft}
            Modern technology and manufacturing advancements has allowed for the
            use of more exotic design decisions that previous heritage aircraft
            could not take advantage of. Regarding airfoil selection, the advent
            of supercritical airfoils has been cited to improve aircraft fuel
            efficiency and thus lower the direct operating cost of the aircraft.
            The Boeing 757 and 767, developed during the 1980s were some of the
            first commercial aircraft to use this technology (2). However, these
            airfoils types are comprised of more complicated compound curves
            which presents added complexity in the manufacturing process,
            driving up initial cost of the aircraft. The initial airframe cost
            increase was not considered in this study.

            The second advanced technology considering during the design process
            herein was the implementation of composites materials for the
            airframe. Airframes have typically been composed of various aluminum
            alloys which presents a strong yet lightweight option. However,
            recent advancements in materials engineering has allowed for
            reliable implementation of composite layups, which are lighter and
            stronger than the aluminum alternative. However, the process to form
            structures from composites is complicated and if not performed
            properly, the composite structures may exhibit delamination.
            Performing the process correctly is expensive and time consuming,
            thus driving up the initial procurement costs of the aircraft using
            this technology. Again, this effect was omitted in this study.
            Figure \ref{fig:weight} plots the takeoff weight of the aircraft
            with different technologies. Dropping the weight of the aircraft
            drops the DOC, thus combination with the lowest weight presents the
            best option.
        \end{flushleft}

%        \begin{center}
%            \includegraphics[scale=0.7]{tech.PNG}
%            \captionof{figure}{Direct operating cost, per ton, per mile for the optimal aircraft configuration using different technologies.}\label{fig:doc}
%        \end{center}

        \begin{center}
            \includegraphics[scale=0.7]{tech_weight.PNG}
            \captionof{figure}{Takeoff Weight for the optimal aircraft configuration using different technologies.}\label{fig:weight}
        \end{center}

        \begin{flushleft}
            It is clear that both technologies, supercritical airfoils and
            composite structures, reduce the direct operating cost. Although
            using a supercritical airfoil may produce a different optimized
            aspect ratio and sweep angle, these effects were not considered in this
            study.
        \end{flushleft}

    \subsection{Aircraft Seat Configuration}
    \label{sec:seat configuration}
        \begin{flushleft}
            The aircraft seat configuration can also be varied in the number of
            aisles that run the length of the fuselage, and the number of seats
            abreast. In this case, the main metric of comparison will be the
            direct operating cost. For this study, the optimized aspect ratio
            and both supercritical airfoil technology and composite technology
            has been applied to the aircraft design. In the previous sections, the
            standard configuration of one aisle and six passengers abreast
            was considered. These results are plotted in figure \ref{fig:seats}.
        \end{flushleft}

        \begin{center}
            \includegraphics[scale=0.7]{seating.PNG}
            \captionof{figure}{Direct operating cost, per ton, per mile for the optimal aircraft configuration using different seating configurations.}\label{fig:seats}
        \end{center}

        \begin{flushleft}
            From this study, it is evident that when holding the number of
            aisles constant, increasing the number of passengers abreast leads
            to a higher direct operating cost. Additionally, it is clear that the
            one aisle, six abreast configuration had a lower direct operating
            cost for both aircraft than any of the two aisle configurations.
            However, with a two aisle configuration, passengers
            are able to load and unload from the aircraft more quickly,
            significantly reducing the down time required when at the airports.
            For aircraft that may be making shorter flights, such as the
            one-stop aircraft in this study, it may be more beneficial to use
            the two-aisle configuration despite the higher direct operating cost
            found in this study. The increased loading and unloading efficiency
            was not taken into account in the direct operating cost estimation
            used in this study.
        \end{flushleft}


    \subsection{Number of Engines}
    \label{sec:engine}
        \begin{flushleft}
            The final design consideration was the number of engines. Figure
            \ref{fig:engine} plots the direct operating cost for a 1 aisle, 6
            abreast airplane using the optimized aspect ratio and wing sweep
            angle, with both advanced technologies applied. It is clear from the
            plot that increasing the number of engines dropped the direct
            operating cost; for the nonstop aircraft, the optimum number of
            engines was three in terms of the direct operating cost. However,
            the final engine count chosen was two for both aircraft as the
            effects of initial procurement and maintenance costs, were not
            considered in the direct operating cost calculation.
        \end{flushleft}

        \begin{center}
            \includegraphics[scale=0.7]{engines.PNG}
            \captionof{figure}{Direct operating cost, per ton, per mile for the optimal aircraft configuration, using 1 aisle and 6 abreast, with different engine count.}\label{fig:engine}%      only if needed
        \end{center}


    %\pagebreak
    \subsection{Final Aircraft Design Specifications}
    \label{sec:optimized}

        \begin{flushleft}
            The table below lists the final optimized specifications of the
            non-stop and one-stop aircraft, as derived from the Python script.
        \end{flushleft}

        \begin{table}[ht]
            \begin{tabular}{|c|c|}
                \hline
                \rowcolor[HTML]{FFC702}
                \multicolumn{2}{|c|}{\cellcolor[HTML]{FFC702}\textbf{Non-stop Aircraft}} \\ \hline
                \textbf{Specification:}            & \textbf{Optimized Value:}           \\ \hline
                Sweep                              & 37 deg                              \\ \hline
                \rowcolor[HTML]{C0C0C0}
                Aspect Ratio                       & 7.9                                 \\ \hline
                \rowcolor[HTML]{FFFFFF}
                Airfoil Type                       & Supercritical                       \\ \hline
                \rowcolor[HTML]{C0C0C0}
                Wing Area                          & 2567 ft$^2$                             \\ \hline
                \rowcolor[HTML]{FFFFFF}
                Wing Span                          & 142 ft                              \\ \hline
                \rowcolor[HTML]{C0C0C0}
                Number of Aisles                   & 1                                   \\ \hline
                \rowcolor[HTML]{FFFFFF}
                Number of Seats Abreast            & 6                                   \\ \hline
                \rowcolor[HTML]{C0C0C0}
                Fuselage Diameter                  & 14.4 ft                             \\ \hline
                \rowcolor[HTML]{FFFFFF}
                Fuselage Length                    & 191 ft                              \\ \hline
                \rowcolor[HTML]{C0C0C0}
                Number of Engines                  & 2                                   \\ \hline
                \rowcolor[HTML]{FFFFFF}
                Structure Type                     & Composite                           \\ \hline
                \rowcolor[HTML]{C0C0C0}
                Takeoff Weight                     & 393945 lbs                          \\ \hline
                \rowcolor[HTML]{FFFFFF}
                Fuel Weight                        & 168068 lbs                           \\ \hline
                \rowcolor[HTML]{C0C0C0}
                DOC                                & 0.078 \$/ton/mile                   \\ \hline
            \end{tabular}
            \quad
            \begin{tabular}{|c|c|}
                \hline
                \rowcolor[HTML]{DAE8FC}
                \multicolumn{2}{|c|}{\cellcolor[HTML]{DAE8FC}\textbf{One-stop Aircraft}} \\ \hline
                \textbf{Specification:}            & \textbf{Optimized Value:}           \\ \hline
                Sweep                              & 31 deg                              \\ \hline
                \rowcolor[HTML]{C0C0C0}
                Aspect Ratio                       & 9.5                                 \\ \hline
                \rowcolor[HTML]{FFFFFF}
                Airfoil Type                       & Supercritical                       \\ \hline
                \rowcolor[HTML]{C0C0C0}
                Wing Area                          & 2224 ft$^2$                             \\ \hline
                \rowcolor[HTML]{FFFFFF}
                Wing Span                          & 145 ft                              \\ \hline
                \rowcolor[HTML]{C0C0C0}
                Number of Aisles                   & 2                                   \\ \hline
                \rowcolor[HTML]{FFFFFF}
                Number of Seats Abreast            & 6                                   \\ \hline
                \rowcolor[HTML]{C0C0C0}
                Fuselage Diameter                  & 14.7 ft                             \\ \hline
                \rowcolor[HTML]{FFFFFF}
                Fuselage Length                    & 174 ft                              \\ \hline
                \rowcolor[HTML]{C0C0C0}
                Number of Engines                  & 2                                   \\ \hline
                \rowcolor[HTML]{FFFFFF}
                Structure Type                     & Composite                           \\ \hline
                \rowcolor[HTML]{C0C0C0}
                Takeoff Weight                     & 241715 lbs                          \\ \hline
                \rowcolor[HTML]{FFFFFF}
                Fuel Weight                        & 64047 lbs                           \\ \hline
                \rowcolor[HTML]{C0C0C0}
                DOC                                & 0.068 \$/ton/mile                   \\ \hline
            \end{tabular}
        \caption{Final aircraft sizing specifications.}
        \end{table}

    \section{Summary and Conclusion}
    \label{sec:conclusion}
        \begin{flushleft}
            This section will summarize the results of the non-stop and one-stop
            aircraft design studies discussed in detail in the previous sections
            herein. Recommendations shall be made as to which aircraft may suit a
            given customer's needs the best.
        \end{flushleft}

    \subsection{Payload Range}
    \label{sec:PR}
    \begin{flushleft}
        This section shows the payload range chart. This chart plots the range
        each optimized aircraft can fly given varying payloads.
    \end{flushleft}

    \begin{center}
        \includegraphics[scale=0.6]{Payload Range.PNG}
        \captionof{figure}{Payload range plotted against the range.}\label{fig:PR}
    \end{center}

    \subsection{Aircraft Configuration Descriptions}
    \label{sec:Configurations}
        \begin{flushleft}
            The previous sections of the report have found the optimized
            specifications of a short haul and long haul aircraft able to meet a
            7400 nautical mile design range, with one stop and non stop flights
            respectively. These specifications are listed in table 2 above. This
            section contains a table with the configuration selections and
            dimensions such as galley size, seat size, window size, and more
            important interior and exterior component dimensions. A
            three-dimensional model of the airplane and corresponding
            two-dimensional drawings were created using SolidWorks. Detailed
            drawings are shown in the last section of the report.
        \end{flushleft}

        \begin{center}
            \includegraphics[scale=0.34]{Final Config.PNG}
            \captionof{figure}{Configuration Parameters}\label{fig:config}
        \end{center}


    \subsection{Recommendations}
    \label{sec:Recommendations}
        \begin{flushleft}
            The final aircraft design specifications were chosen based on
            optimizing the minimum direct operating cost, providing appealing
            options for the customers needs, and creating a safe aircraft. Both
            aircraft utilized two engines. The one-stop aircraft utilized only a
            two aisle, six abreast seating configuration while the non-stop
            aircraft used a one aisle, six abreast seating configuration. The
            two aisle configuration was chosen for the smaller airplane for its
            ability to load and unload passengers more quickly, thus decreasing
            the idle time at airports and increasing the maximum number of
            flights than may be conducted on a given day, which could reduce the DOC.


            The one-stop flight presents the better option over the non-stop
            flight. Not only does the one-stop flight have a shorter operating
            cost per ton, per mile; but, the one-stop flight also uses less fuel
            (when doubled) than the non-stop flight. The fuel burn for the
            non-stop aircraft was 168,068 pounds while two one-stop flights to
            achieve the 7,400 nmi range only used 128,094. In total, the
            one-stop aircraft saves roughly 24\% fuel, and thus carbon
            emissions, compared to the non-stop aircraft. The one-stop flight
            presents the superior option to the non-stop aircraft in all
            considerations - cost to the airline, cost to the passenger, and
            cost to the environment.
        \end{flushleft}

    %\pagebreak
    \section{References}
        \begin{flushleft}
            \begin{enumerate}
                \item Hall, Nancy. (2018). Wing Geometry Definitions. NASA Glenn Research Center.
                \item Roeseler, W. G., et al. (2007). Composite Structures: The First 100 Years. 16TH INTERNATIONAL CONFERENCE ON COMPOSITE MATERIALS.
                \item Schaufele, R. D. (2007). The Elements of Aircraft Preliminary Design. Santa Ana, CA: ARIES Publication.
                \item Shevell, R. S. (1983). Fundamentals of flight. Englewood Cliffs, NJ: Prentice-Hall.
            \end{enumerate}
        \end{flushleft}

    \section{Configurations}
    \begin{flushleft}
        The following pages show a three-view, layout of passenger arrangement,
        and a wing-tail diagram for both optimized aircraft. The drawings are
        presented on size B paper and all dimensions are shown to the nearest
        tenth of an inch.
    \end{flushleft}

\end{document}
